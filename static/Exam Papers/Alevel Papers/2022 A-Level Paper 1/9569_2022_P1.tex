%% LyX 2.3.6.1 created this file.  For more info, see http://www.lyx.org/.
%% Do not edit unless you really know what you are doing.
\documentclass{article}
\usepackage[latin9]{inputenc}
\usepackage{geometry}
\geometry{verbose,tmargin=1cm,bmargin=1cm,lmargin=1cm,rmargin=1cm}
\usepackage{setspace}

\makeatletter
%%%%%%%%%%%%%%%%%%%%%%%%%%%%%% User specified LaTeX commands.
\usepackage[T1]{fontenc}
\usepackage[latin9]{inputenc}
\usepackage{color}
\usepackage{amsmath}
\usepackage{amsthm}
\usepackage{setspace}
\PassOptionsToPackage{normalem}{ulem}
\usepackage{ulem}

\makeatletter

\providecommand{\tabularnewline}{\\}

\numberwithin{equation}{section}
\numberwithin{figure}{section}

\usepackage{helvet}
\renewcommand{\familydefault}{\sfdefault}
\usepackage[T1]{fontenc}
\usepackage[latin9]{inputenc}
\usepackage{geometry}
\geometry{verbose,tmargin=1.8cm,bmargin=4cm,lmargin=1.5cm,rmargin=2cm}
\usepackage{enumitem}
\usepackage{amstext}
\usepackage{amsthm}
\usepackage{amssymb}
\usepackage{setspace}
\usepackage{graphicx}
\doublespacing


\usepackage{enumitem}
\setenumerate[1]{label=\textbf{\arabic*}}
\setenumerate[2]{label=\textbf{(\alph*)}}
\setenumerate[3]{label=\textbf{(\roman*)}}
\setlist[enumerate]{align=right}

\setcounter{page}{2}

\makeatother

\usepackage{babel}

\makeatother

\begin{document}
\begin{enumerate}
\item A league of cycle clubs has competitive events throughout the year.
Each event is scored. The club with the highest score at the end of
the year is declared the league champion.

A database is to be used to store the data required about clubs and
events.

Each club has a unique club name. The name and email address of the
club secretary is stored.

Each event has a unique event number. The date. start time and location
of each event is stored.

Each event is limited to a maximum of 175 cyclists with a maximum
of seven cyclists from any club. When a cyclist enters an event their
name and club name are recorded, and they are assigned a unique competitor
number in the range 1 to 175 inclusive.

When an event is over. the finishing position and finishing time is
stored for each competing cyclist.

The event organiser calculates a score using the positions and times
of each cyclist. The score is a value in the range 0 to 25 inclusive
and is awarded to each participating club. The club name.

the event number and the club score are recorded.

\textbf{Four} entities are to be used to define the data needs of
the league. The database has \textbf{not} been normalised.
\begin{enumerate}
\item Copy and complete this entity-relationship (ER) diagram by showing
\textbf{four} many-to-one relationships. When copying the diagram,
do \textbf{not} rearrange the layout of the four given entities. If
needed. your answer may include lines that cross over each other.

\textless INSERT\_IMAGE\_HERE\textgreater{} \hfill{}{[}4{]}
\item Write table definitions, indicating the primary key. for each of the
tables listed below. 

Use the format: \texttt{TableName (Attributel , Attribute2, Attribute3,
etc .)} 
\begin{enumerate}
\item Club\hfill{} {[}2{]}
\item Event \hfill{}{[}2{]}
\item EventCompetitor \hfill{}{[}3{]}
\item EventClubScore\hfill{} {[}3{]}
\end{enumerate}
\item After each event a report is produced displaying. for all competing
clubs: the club's name. the club\textquoteright s score for the event.
the secretary's name and the secretary's email address. 

Write an SQL query that will output the required data for an event
with the event number 23. The output must list the clubs in the order
of highest score to lowest score. {[}6{]} 
\item Normalisation is a process used when designing database tables. 
\begin{enumerate}
\item State \textbf{two} aims of the normalisation process. \hfill{}{[}2{]}
\end{enumerate}
Assume that a table is already in first normal form (1NF). 
\begin{enumerate}
\item[\textbf{(ii)}] State \textbf{two} other requirements of a table being in third normal
form (3NF).\hfill{} {[}2{]}
\end{enumerate}
\item Identify the most suitable validation technique for an event score.\hfill{}
{[}1{]}
\end{enumerate}
\item An estate agent maintains a list of apartments for sale or rent. They
want to use Object-Oriented Programming (OOP) to model this situation. 

For every apartment the following data is recorded: 
\begin{itemize}
\item the apartment address 
\item the owner's name. address and email address. 
\end{itemize}
For an apartment that is for sale the following data is recorded: 
\begin{itemize}
\item the asking price
\item the date of sale
\item the price paid. 
\end{itemize}
Until an apartment is sold. the date of sale and price paid are left
blank. 

For a rental apartment the following data is recorded: 
\begin{itemize}
\item monthly rent 
\item date when rent is to be paid.
\end{itemize}
If a rental apartment is vacant, the rent date (date when rent is
to be paid) is set to 31/12/2099. When the rent is paid. the rent
date is updated. 
\begin{enumerate}
\item Explain the difference between a class and an object. \hfill{}{[}2{]}
\item Draw a class diagram for the described situation. showing:
\begin{itemize}
\item any derived classes and inheritance from the base class
\item the properties needed in the base and any derived classes 
\item suitable methods. in each class. to support the system. \hfill{}{[}8{]}
\end{itemize}
\end{enumerate}
It is common for the properties of a class to be private. 
\begin{enumerate}
\item[\textbf{(c)}] Explain an advantage of using private properties. \hfill{}{[}2{]}
\item[\textbf{(d)}] Explain a benefit of using inheritance to a software developer. \hfill{}{[}2{]}
\end{enumerate}
\item Messages are sent across a public network. 
\begin{enumerate}
\item State how a message can be made meaningless to anyone other than the
intended recipient. \hfill{}{[}1{]}
\item Explain what sending and receiving devices can do to detect any malicious
alteration of a message. \hfill{}{[}4{]}
\end{enumerate}
Authentication is used in a computing context. 
\begin{enumerate}
\item[\textbf{(c)}] Describe \textbf{two} situations where authentication is important.
For each situation. state how it can be achieved. \hfill{}{[}4{]}
\end{enumerate}
\item A program needs a structure to store ordered data. Data items are
added and removed from the structure during the execution of the program.
The programmer is considering using either a fixed- capacity array
or a linked list to hold the ordered data items.
\begin{enumerate}
\item State \textbf{two} advantages of using a fixed---capacity array over
a linked list to store the ordered items. \hfill{}{[}2{]}
\item State \textbf{two} advantages of using a linked list over a fixed-capacity
array to store the ordered items. \hfill{}{[}2{]}
\item A linked list can be represented as a number of nodes each containing
an item of data and a pointer to the next node. A pointer \texttt{Head}
indicates the location of the first node. 

A value of \texttt{-1} in \texttt{Head} indicates an empty list; a
value of \texttt{-1} in a pointer indicates the final node.

\textless INSERT\_IMAGE\_HERE\textgreater{} 

Function \texttt{z} is written to operate on a linked list \texttt{LL}.
The function has a single integer parameter, \texttt{CurrentPointer}.
The function returns an integer. 

\noindent\begin{minipage}[t]{1\columnwidth}%
\begin{singlespace}
\noindent \texttt{01 FUNCTION Z (CurrentPointer) RETURNS INTEGER}

\noindent \texttt{02 ~~IF CurrentPointer = -1 THEN}

\noindent \texttt{03 ~~~~RETURN 0}

\noindent \texttt{04 ~~ELSE}

\noindent \texttt{05 ~~~~CurrentPointer = LL(CurrentPointer).Pointer}

\noindent \texttt{O6 ~~~~RETURN l + Z(CurrentPointer)}

\noindent \texttt{07 ~~ENDIF}

\noindent \texttt{08 ENDFUNCTION}
\end{singlespace}

\bigskip{}
%
\end{minipage}
\begin{enumerate}
\item State what line \texttt{06} indicates about function \texttt{z} \hfill{}{[}1{]} 
\item State what lines \texttt{02} and \texttt{03} represent. \hfill{}{[}1{]} 
\item State the purpose of function \texttt{z} \hfill{}{[}1{]} 
\end{enumerate}
\item Using pseudo-code write a function. of the same type as function \texttt{z},
to reverse the order of the data items in the linked list \texttt{LL}.
You can assume that \texttt{Head} points to the first item in the
current list. \hfill{} {[}5{]}
\item It is common to sort items in a fixed-capacity array. 

Explain why a merge sort may be faster than a quicksort in this situation.
\hfill{}{[}2{]}
\end{enumerate}
\item A programmer is writing code for a system to control an elevator.
The programming language used allows the use of dynamic memory allocation. 

The following algorithm describes what is to happen after a floor
button in the elevator has been pressed: 

Get some memory that will be used to store the chosen floor number 

Store the chosen floor number into the memory 

Is the elevator already at the chosen floor? 

\texttt{~~}Yes: 

\texttt{~~~~}Finished 

\texttt{~~}No:

\texttt{~~~~}Close the doors

\texttt{~~~~}Move to the required floor

\texttt{~~~~}Open the doors

\texttt{~~~~}Release the memory used to store the floor number 

You may assume that only one button press is processed at a time. 

The programmer believes that they have thoroughly tested the complete
system and finds it to operate exactly as expected. 

After many months in operation the elevator ceases to operate. The
engineer called to investigate the fault switches off the power to
the elevator and control system. finds no obvious problem. switches
the power back on and finds that the elevator operates correctly.
Several months later the elevator ceases to operate again. 
\begin{enumerate}
\item Explain how a mistake in the algorithm is causing the recurring problem.
\hfill{}{[}3{]}
\item Explain how the algorithm should be changed to prevent the problem
recurring. \hfill{}{[}2{]}
\item Suggest \textbf{two} reasons why the mistake was \textbf{not} identified
during testing. \hfill{}{[}2{]}
\end{enumerate}
\item Every night a business backs up all its data. From time to time the
data that is infrequently accessed is archived.
\begin{enumerate}
\item {}
\begin{enumerate}
\item Describe the purpose of creating a backup. \hfill{}{[}2{]}
\item Describe the purpose of archiving. \hfill{}{[}2{]}
\end{enumerate}
\item Explain why backup copies of the data should be stored off-site. \hfill{}{[}2{]}
\item Describe the consequences of the business \textbf{not} backing up
the data. \hfill{}{[}3{]}
\end{enumerate}
\item Data transmitted across the intemet is divided into sequentially numbered
packets. 
\begin{enumerate}
\item Explain why transmitted data is divided into packets. \hfill{}{[}2{]}
\begin{enumerate}
\item Explain why the packets are sequentially numbered. \hfill{}{[}2{]}
\item State \textbf{two} items, other than the packet number, that are stored
in the packet header. \hfill{}{[}2{]}
\end{enumerate}
\item Explain why protocols are required to enable reliable communication
over the intemet. \hfill{}{[}2{]}
\item A router is a device that allows the connection of a LAN to the lnternet.

Explain how the router directs arriving data packets to the comet
device on the LAN. \hfill{}{[}2{]}
\item A firewall is often placed between a LAN and the Internet. 

Explain how a firewall can provide security to the LAN.\hfill{} {[}2{]}
\end{enumerate}
\item A hash function and associated hash table are commonly used when finding
storage space for a new record and searching for a specific record
within a data set. 
\begin{enumerate}
\item Explain the advantage a hash table search might have over a linear
search and a binary search when searching for a specific record. Refer
to time complexity in your answer. \hfill{}{[}3{]}
\item Explain the meaning of a collision in the context of a hash table
search. \hfill{}{[}2{]}
\item Describe \textbf{one} method that can be used to handle the consequence
of a collision. \hfill{}{[}2{]}
\item A hashing algorithm is used to calculate the index of a hash table
from a record key. Give \textbf{three} features of an effective hashing
algorithm. \hfill{}{[}3{]}
\end{enumerate}
\end{enumerate}

\end{document}
