%% LyX 2.3.6.1 created this file.  For more info, see http://www.lyx.org/.
%% Do not edit unless you really know what you are doing.
\documentclass{article}
\usepackage[latin9]{inputenc}
\usepackage{geometry}
\geometry{verbose,tmargin=1cm,bmargin=1cm,lmargin=1cm,rmargin=1cm}
\usepackage{setspace}

\makeatletter

%%%%%%%%%%%%%%%%%%%%%%%%%%%%%% LyX specific LaTeX commands.
%% Because html converters don't know tabularnewline
\providecommand{\tabularnewline}{\\}

%%%%%%%%%%%%%%%%%%%%%%%%%%%%%% User specified LaTeX commands.
\usepackage[T1]{fontenc}
\usepackage[latin9]{inputenc}
\usepackage{color}
\usepackage{amsmath}
\usepackage{amsthm}
\usepackage{setspace}
\PassOptionsToPackage{normalem}{ulem}
\usepackage{ulem}

\makeatletter

\providecommand{\tabularnewline}{\\}

\numberwithin{equation}{section}
\numberwithin{figure}{section}

\usepackage{helvet}
\renewcommand{\familydefault}{\sfdefault}
\usepackage[T1]{fontenc}
\usepackage[latin9]{inputenc}
\usepackage{geometry}
\geometry{verbose,tmargin=1.8cm,bmargin=4cm,lmargin=1.5cm,rmargin=2cm}
\usepackage{enumitem}
\usepackage{amstext}
\usepackage{amsthm}
\usepackage{amssymb}
\usepackage{setspace}
\usepackage{graphicx}
\doublespacing


\usepackage{enumitem}
\setenumerate[1]{label=\textbf{\arabic*}}
\setenumerate[2]{label=\textbf{(\alph*)}}
\setenumerate[3]{label=\textbf{(\roman*)}}
\setlist[enumerate]{align=right}

\setcounter{page}{2}

\makeatother

\usepackage{babel}

\makeatother

\begin{document}
\begin{enumerate}
\item Name your Jupyter Notebook as:

\texttt{TASKl\_\textless your name\textgreater\_\textless centre
number\textgreater\textasciitilde\textless index number\textgreater .ipynb}

The task is to implement a Caesar cypher encryption algorithm.

A Caesar cypher encodes each letter with a different letter. A 10-place
Caesar cypher uses the ASCII value of each letter and adds the number
10 to it.

For example:
\begin{itemize}
\item The letter \textquoteleft A' has the ASCII value 65. Adding 10 to
the number will give the ASCII value 75. The character for 75 is 'K'.
\item When an uppercase letter's code goes beyond \textquoteleft 2' it returns
to \textquoteleft A'. For example. the character \textquoteleft 2'
will be encrypted as 'J'.
\item When a lowercase letter's code goes beyond \textquoteleft 2' it returns
to \textquoteleft a'. For example. the character 'x' will be encrypted
as \textquoteleft h'.
\item Spaces ' ' are replaced with the character 'I'.
\end{itemize}
You will only need to convert letters and spaces. If the character
is invalid. -1 is returned.

The following table shows the ASCII values of some of the characters.
\noindent \begin{center}
\begin{tabular}{|c|c|}
\hline 
\textbf{Character} & \textbf{ASCII value}\tabularnewline
\hline 
\hline 
A & 65\tabularnewline
\hline 
Z & 90\tabularnewline
\hline 
a & 97\tabularnewline
\hline 
z & 122\tabularnewline
\hline 
' ' (space) & 32\tabularnewline
\hline 
! & 33\tabularnewline
\hline 
\end{tabular}
\par\end{center}

In ASCII the letters follow on numerically. For example. the letter
'A' is 65, \textquoteleft B\textquoteleft{} is 66. \textquoteleft C'
Is 67 etc. '

For each of the sub-tasks, add a comment statement at the beginning
of the code using the hash symbol \textquoteleft \#' to indicate the
sub-task the program code belongs to, for example:

\begin{singlespace}
\noindent \texttt{}%
\begin{tabular}{c|lcccccccccccccccccccccc|}
\cline{2-24} \cline{3-24} \cline{4-24} \cline{5-24} \cline{6-24} \cline{7-24} \cline{8-24} \cline{9-24} \cline{10-24} \cline{11-24} \cline{12-24} \cline{13-24} \cline{14-24} \cline{15-24} \cline{16-24} \cline{17-24} \cline{18-24} \cline{19-24} \cline{20-24} \cline{21-24} \cline{22-24} \cline{23-24} \cline{24-24} 
\texttt{In {[}1{]} :} & \texttt{\#Task 1.1} &  &  &  &  &  &  &  &  &  &  &  &  &  &  &  &  &  &  &  &  &  & \tabularnewline
 & \texttt{Program Code} &  &  &  &  &  &  &  &  &  &  &  &  &  &  &  &  &  &  &  &  &  & \tabularnewline
\cline{2-24} \cline{3-24} \cline{4-24} \cline{5-24} \cline{6-24} \cline{7-24} \cline{8-24} \cline{9-24} \cline{10-24} \cline{11-24} \cline{12-24} \cline{13-24} \cline{14-24} \cline{15-24} \cline{16-24} \cline{17-24} \cline{18-24} \cline{19-24} \cline{20-24} \cline{21-24} \cline{22-24} \cline{23-24} \cline{24-24} 
\multicolumn{1}{c}{} & \texttt{Output:} &  &  &  &  &  &  &  &  &  &  &  &  &  &  &  &  &  &  &  &  &  & \multicolumn{1}{c}{}\tabularnewline
\end{tabular}
\end{singlespace}

\subsection*{Task 1.1}

This program will encrypt each letter by adding the number 10 to its
ASCII value. Spaces will be replaced with T. If the character Is invalid,
-1 is returned.

Write program code for a function that takes a character as a parameter
and returns the ASCII value of the new encrypted character.\hfill{}
{[}5{]}

\subsection*{Task 1.2}

Write program code to:
\begin{itemize}
\item read in a single character from the user
\item call your function from \textbf{Task 1.1} with this character
\item output its encrypted character or an appropriate message if the character
is invalid. \hfill{}{[}3{]}
\end{itemize}
Test your program four times. with the following test data:

\noindent\begin{minipage}[t]{1\columnwidth}%
\texttt{A}

\texttt{a}

\texttt{\#}

\texttt{' '} (space)%
\end{minipage}

\hfill{}{[}2{]}

\subsection*{Task 1.3}

The text file\texttt{ DATATOENCRYPT.txt} contains a message that needs
to be encrypted and then stored in a text file named \texttt{ENCRYPTEDMESSAGE.txt}.

Write program code to:
\begin{itemize}
\item read the data from the text file \texttt{DATATOENCRYPT.txt}
\item use your function from \textbf{Task 1.1} to encrypt each character
\item store the encrypted message in the text file \texttt{ENCRYPTEDMESSAGE.txt}
\end{itemize}
Do not append invalid characters to the encrypted message. \hfill{}{[}7{]}

Test your program with DATATOENCRYPT.txt

Show the contents of EI'ICRYPTEDMESSAGE. txt after you have run the
program.\hfill{} {[}1{]}

Save your Jupyter Notebook for Task 1.
\item Name your Jupyter Notebook as: 

\texttt{TASK2\_\textless your name\textgreater\_\textless centre
number\textgreater\_\textless index number\textgreater .ipynb }

This task is to perform sorting algorithms on 100 integers held in
a 1-dimensional list and then search for a value in the list. 

For each of the sub-tasks, add a comment statement at the beginning
of the code using the hash symbol \textquoteleft \#' to indicate the
sub-task the program code belongs to, for example:

\begin{singlespace}
\noindent \texttt{}%
\begin{tabular}{c|lcccccccccccccccccccccc|}
\cline{2-24} \cline{3-24} \cline{4-24} \cline{5-24} \cline{6-24} \cline{7-24} \cline{8-24} \cline{9-24} \cline{10-24} \cline{11-24} \cline{12-24} \cline{13-24} \cline{14-24} \cline{15-24} \cline{16-24} \cline{17-24} \cline{18-24} \cline{19-24} \cline{20-24} \cline{21-24} \cline{22-24} \cline{23-24} \cline{24-24} 
\texttt{In {[}1{]} :} & \texttt{\#Task 2.1} &  &  &  &  &  &  &  &  &  &  &  &  &  &  &  &  &  &  &  &  &  & \tabularnewline
 & \texttt{Program Code} &  &  &  &  &  &  &  &  &  &  &  &  &  &  &  &  &  &  &  &  &  & \tabularnewline
\cline{2-24} \cline{3-24} \cline{4-24} \cline{5-24} \cline{6-24} \cline{7-24} \cline{8-24} \cline{9-24} \cline{10-24} \cline{11-24} \cline{12-24} \cline{13-24} \cline{14-24} \cline{15-24} \cline{16-24} \cline{17-24} \cline{18-24} \cline{19-24} \cline{20-24} \cline{21-24} \cline{22-24} \cline{23-24} \cline{24-24} 
\multicolumn{1}{c}{} & \texttt{Output:} &  &  &  &  &  &  &  &  &  &  &  &  &  &  &  &  &  &  &  &  &  & \multicolumn{1}{c}{}\tabularnewline
\end{tabular}
\end{singlespace}

\subsection*{Task 2.1 }

Write a function, \texttt{task2\_1()} to: 
\begin{itemize}
\item initialise a global 1-dimensional list 
\item generate 100 random integers between 1 and 100 (inclusive)
\item store each Integer in the list 
\item output the contents of the list.\hfill{} {[}2{]}
\item Test your program and show the output. \hfill{}{[}1{]}
\end{itemize}

\subsection*{Task 2.2 }

Write program code to: 
\begin{itemize}
\item declare a procedure to implement a bubble sort 
\item implement a bubble sort to sort the unsorted 1-dimensional list from
\textbf{Task 2.1} (you must \textbf{not} use a built-in function). 
\end{itemize}
Write a program to: 
\begin{itemize}
\item call your function from \textbf{Task 2.1}
\item call the bubble sort procedure
\item output the sorted list. \hfill{}{[}5{]}
\end{itemize}
Test your program and show the output. \hfill{}{[}1{]}

\subsection*{Task 2.3}

Write program code to:
\begin{itemize}
\item declare a procedure to implement a merge sort
\item implement a merge sort on the unsorted 1-dimensional list from Task
2.1 (you must not use a built-in function).
\end{itemize}
Write a program to:
\begin{itemize}
\item ask the user to select which sorting algorithm they want to use
\item loop until a valid choice is input
\item call your function from Task 2.1
\item call the appropriate sorting procedure
\item output the sorted list. \hfill{}{[}7{]}
\end{itemize}
Test the program by first entering one invalid choice (for example,
'no') and then by entering one of the sorting methods (for example.
'bubble').\hfill{} {[}1{]}

\subsection*{Task 2.4}

Write program code to declare a recursive function to:
\begin{itemize}
\item take an integer value as one of its parameters
\item implement a recursive blnary search on the list from \textbf{Task
2.1}
\item return the position of the integer in the list (if the integer appears
more than once. then only one position needs to be returned)
\item return -1 if the integer is \textbf{not} in the list.
\end{itemize}
Write a program to:
\begin{itemize}
\item read in an integer value from the user
\item call the binary search function with the integer input
\item output the returned position if the integer is in the list
\item output \textquotedblleft Not found' if the integer is \textbf{not}
in the list. \hfill{}{[}7{]}
\end{itemize}
Test the program with the smallest integer that is in the list and
then with one integer that is \textbf{not} in the list.\hfill{} {[}2{]}

Save your Jupyter Notebook for Task 2.
\item Name your Jupyter Notebook as: 

\texttt{TASK3\_\textless your name\textgreater\_\textless centre
number\textgreater\_\textless index number\textgreater .ipynb }

A binary search tree is used to store 10 integer values between 0
and 999 (inclusive) in ascending numerical order. 

The \texttt{Tree} is implemented using Object-Oriented Programming
(OOP). 

The class \texttt{Tree} contains three properties: 
\begin{itemize}
\item \texttt{left\_pointer} points to the left subtree 
\item \texttt{right\_pointer} points to the right subtree 
\item \texttt{data} is the data in the node. 
\end{itemize}
The class \texttt{Tree} contains the following methods: 
\begin{itemize}
\item a constructor to set the left pointer and right pointer to None. and
the data to its parameter 
\item a recursive method to take the parameter and store it in the correct
position in the tree
\item a recursive method to use in-order traversal to output the data in
the tree 
\item a recursive method to use post-order traversal to output the data
in the tree. 
\end{itemize}
For the sub-task, add a comment statement at the beginning of the
code using the hash symbol \textquoteleft \#' to indicate the sub-task
the program code belongs to, for example:

\begin{singlespace}
\noindent \texttt{}%
\begin{tabular}{c|lcccccccccccccccccccccc|}
\cline{2-24} \cline{3-24} \cline{4-24} \cline{5-24} \cline{6-24} \cline{7-24} \cline{8-24} \cline{9-24} \cline{10-24} \cline{11-24} \cline{12-24} \cline{13-24} \cline{14-24} \cline{15-24} \cline{16-24} \cline{17-24} \cline{18-24} \cline{19-24} \cline{20-24} \cline{21-24} \cline{22-24} \cline{23-24} \cline{24-24} 
\texttt{In {[}1{]} :} & \texttt{\#Task 3.1} &  &  &  &  &  &  &  &  &  &  &  &  &  &  &  &  &  &  &  &  &  & \tabularnewline
 & \texttt{Program Code} &  &  &  &  &  &  &  &  &  &  &  &  &  &  &  &  &  &  &  &  &  & \tabularnewline
\cline{2-24} \cline{3-24} \cline{4-24} \cline{5-24} \cline{6-24} \cline{7-24} \cline{8-24} \cline{9-24} \cline{10-24} \cline{11-24} \cline{12-24} \cline{13-24} \cline{14-24} \cline{15-24} \cline{16-24} \cline{17-24} \cline{18-24} \cline{19-24} \cline{20-24} \cline{21-24} \cline{22-24} \cline{23-24} \cline{24-24} 
\multicolumn{1}{c}{} & \texttt{Output:} &  &  &  &  &  &  &  &  &  &  &  &  &  &  &  &  &  &  &  &  &  & \multicolumn{1}{c}{}\tabularnewline
\end{tabular}
\end{singlespace}

\subsection*{Task 3.1}

Write program code to declare the class Tree and its constructor.
\hfill{}{[}4{]}

Write the recursive method to insert a new node into the tree. \hfill{}{[}6{]}

Write the main program to: 
\begin{itemize}
\item declare a new instance of \texttt{Tree} 
\item generate 10 unique random integer values between 0 and 999 (inclusive) 
\item store each unique value as a new node in the tree using your method.
\hfill{}{[}5{]}
\end{itemize}
Write program code to:
\begin{itemize}
\item declare the method to output the in-order traversal of the binary
tree
\item declare the method to output the post-order traversal of the binary
tree.
\end{itemize}
Call the in-order and post-order methods using your tree structure.
\hfill{}{[}7{]}

Test your program and show the output from each traversal. \hfill{}{[}2{]}

Save your Jupyter Notebook for Task 3.
\item Name your Jupyter Notebook as: 

\texttt{TASK4\_\textless your name\textgreater\_\textless centrc
number\textgreater\_\textless index number\textgreater .ipynb }

A library currently keeps paper records about its members. books and
the books loaned. The library wants to create a suitable database
to store the data and to allow them to run searches for specific data.
The database will have three tables: a table to store data about the
books. a table about the members and a table about the loans. The
fields in each table are: 

\texttt{Book}: 
\begin{itemize}
\item \texttt{BookID} - unique book number, for example, 1234 
\item \texttt{Title} - the book title 
\item \texttt{Genre} --- the type of book. for example. Drama. Sci-ii.
Classic. 
\end{itemize}
\texttt{Member}: 
\begin{itemize}
\item \texttt{MemberNumber} - member's unique number, for example, 634 
\item \texttt{FamilyName} - member\textquoteleft s family name 
\item \texttt{GivenName} - member's given name. 
\end{itemize}
\texttt{Loan}: 
\begin{itemize}
\item \texttt{LoanID} - the loan's unique number, for example, 12 
\item \texttt{MemberNumber} - the member's unique number 
\item \texttt{BookID} - the unique book number
\item \texttt{DateLoaned} - the date that the book was taken out by the
member 
\item \texttt{Returned} - \texttt{TRUE} if the book has been returned, or
\texttt{FALSE} if it has not been returned. 
\end{itemize}
For each of the sub-tasks 4.1 to 4.3, add a comment statement at the
beginning of the code using the hash symbol \textquoteleft \#' to
indicate the sub-task the program code belongs to, for example:

\begin{singlespace}
\noindent \texttt{}%
\begin{tabular}{c|lcccccccccccccccccccccc|}
\cline{2-24} \cline{3-24} \cline{4-24} \cline{5-24} \cline{6-24} \cline{7-24} \cline{8-24} \cline{9-24} \cline{10-24} \cline{11-24} \cline{12-24} \cline{13-24} \cline{14-24} \cline{15-24} \cline{16-24} \cline{17-24} \cline{18-24} \cline{19-24} \cline{20-24} \cline{21-24} \cline{22-24} \cline{23-24} \cline{24-24} 
\texttt{In {[}1{]} :} & \texttt{\#Task 4.1} &  &  &  &  &  &  &  &  &  &  &  &  &  &  &  &  &  &  &  &  &  & \tabularnewline
 & \texttt{Program Code} &  &  &  &  &  &  &  &  &  &  &  &  &  &  &  &  &  &  &  &  &  & \tabularnewline
\cline{2-24} \cline{3-24} \cline{4-24} \cline{5-24} \cline{6-24} \cline{7-24} \cline{8-24} \cline{9-24} \cline{10-24} \cline{11-24} \cline{12-24} \cline{13-24} \cline{14-24} \cline{15-24} \cline{16-24} \cline{17-24} \cline{18-24} \cline{19-24} \cline{20-24} \cline{21-24} \cline{22-24} \cline{23-24} \cline{24-24} 
\multicolumn{1}{c}{} & \texttt{Output:} &  &  &  &  &  &  &  &  &  &  &  &  &  &  &  &  &  &  &  &  &  & \multicolumn{1}{c}{}\tabularnewline
\end{tabular}
\end{singlespace}

\subsection*{Task 4.1}

Write a Python program that uses SQL code to create the database \texttt{LIBRARY}
with the three tables given. Define the primary and foreign keys for
each table. \hfill{}{[}6{]} 

\subsection*{Task 4.2}

The text files \texttt{BOOK.txt}, \texttt{MEMBER.txt} and \texttt{LOAN.txt};
store the comma-separated values for each of the tables in the database.

Write a Python program to read in the data from each file and then
store each item of data in the correct place in the database. \hfill{}{[}5{]}

\subsection*{Task 4.3}

Write a Python program to input a member\textquoteleft s number and
return the names of all the books that they have had out on loan,
and whether each book has been returned. \hfill{}{[}5{]}

Test your program by running the application with the member number
200 \hfill{}{[}2{]}

Save your Jupyter Notebook.

\subsection*{Task 4.4}

Write a Python program and the necessary files to create a web application,
that displays the following data, about books that have \textbf{not}
yet been returned:
\begin{itemize}
\item member's family name
\item member\textquoteleft s given name
\item book title.
\end{itemize}
The program should return an HTML document that enables the web browser
to display a table with the required data.

Save your Python program as:

\texttt{TASK\_4\_4\_\textless your name\textgreater\_\textless centre
number\textgreater\_\textless index number\textgreater .py}

with any additional files / subfolders in a folder named:

\texttt{TASK\_4\_4\_\textless your name\textgreater\_\textless centre
number\textgreater\_\textless index number\textgreater} \hfill{}
{[}6{]}

Run the web application.

Save the webpage output as:

\texttt{TASK\_4\_4\_\textless your name\textgreater\_\textless centre
number\textgreater\_\textless index number\textgreater .htm1}
\hfill{}{[}2{]}
\end{enumerate}

\end{document}
